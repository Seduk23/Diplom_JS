\section{Техническое задание}
\subsection{Основание для разработки}

Основанием для разработки является задание на выпускную квалификационную работу бакалавра "<Веб-приложение для компьютерной поддержки самостоятельной работы  иностранных студентов при изучении языка программирования  JavaScript">.

\subsection{Цель и назначение разработки}

Основной задачей выпускной квалификационной работы является разработка и внедрение веб-приложения для компьютерной поддержки самостоятельной работы  иностранных студентов при изучении языка программирования  JavaScript.

Посредством внедрения веб-приложения планируется устранить существующие недостатки, связанные с неструктурированным доступом к учебным материалам, отсутствием интерактивных инструментов для практики и тестирования, а также сложностями в организации учебного процесса для иностранных студентов, включая языковые барьеры.

Цель разработки включает следующие подцели:

\begin{itemize}
\item создание единой образовательной платформы для доступа к учебным курсам, урокам и тестам;
\item обеспечение удобного и интуитивно понятного интерфейса для самостоятельного изучения JavaScript;
\item интеграция инструментов для проверки знаний, таких как тесты с автоматической оценкой;
\item оптимизация процессов управления учебным контентом для преподавателей и взаимодействия студентов с платформой.
\end{itemize}

\subsection{Функциональные задачи}

Разрабатываемая веб-платформа включает в себя следующие модули:
\begin{itemize}
\item \textbf{Курсы} — модуль для создания и управления учебными курсами, включающими уроки и тесты. Преподаватели могут добавлять, редактировать и удалять курсы, а студенты получают доступ к материалам.;
\item \textbf{Уроки} — система управления учебным контентом, позволяющая структурировать материалы курса (текст, изображения, видео) и задавать порядок уроков. Поддерживается предпросмотр уроков и их редактирование.;
\item \textbf{Тесты} — модуль для создания и прохождения интерактивных тестов. Преподаватели могут задавать вопросы и варианты ответов, а студенты проходят тесты с автоматической оценкой результатов;
\item \textbf{Результаты тестов} — инструмент для анализа успеваемости студентов. Преподаватели могут просматривать, удалять и управлять результатами тестов, включая статистику по студентам;
\item \textbf{Панель управления преподавателя} — интерфейс для управления курсами, уроками, тестами и результатами, с удобной навигацией и виджетами для быстрого доступа;
\item \textbf{Профиль пользователя} — модуль для управления учетной записью, включая настройки аватара и персональной информации;
\item \textbf{Сообщения} — система уведомлений для обратной связи (например, сообщения об успешном добавлении урока или удалении результатов);
\item \textbf{Панель управления} — страница с виджетами всех вышеперечисленных сервисов.
\end{itemize}

\subsection{Требования пользователя к интерфейсу web-сайта}

Платформа должна обеспечивать:
\begin{itemize}
    \item авторизацию;
    \item интуитивно понятную навигацию между модулями;
    \item адаптивный интерфейс для десктопных и мобильных устройств.
\end{itemize}

Композиция интерфейса пользователя представлена на рисунках ~\ref{templ:image1}, ~\ref{templ:image2}, ~\ref{templ:image3}, ~\ref{templ:image4}, ~\ref{templ:image5}, ~\ref{templ:image6}, ~\ref{templ:image7}, ~\ref{templ:image8}, ~\ref{templ:image9},  ~\ref{templ:image10}.

\begin{figure}[ht]
	\centering
	\includegraphics[width=1\linewidth]{images/курсы}
	\caption{Композиция интерфейса сервиса <<Курсы>>}
	\label{templ:image1}
\end{figure}

\begin{figure}[ht]
	\centering
	\includegraphics[width=1\linewidth]{images/учитель}
	\caption{Композиция интерфейса сервиса <<Панель управления преподавателя>>}
	\label{templ:image2}
\end{figure}

\begin{figure}[ht]
	\centering
	\includegraphics[width=1\linewidth]{images/уроки}
	\caption{Композиция интерфейса сервиса <<Уроки>>}
	\label{templ:image3}
\end{figure}

\begin{figure}[ht]
	\centering
	\includegraphics[width=1\linewidth]{images/Тесты}
	\caption{Композиция интерфейса сервиса <<Тесты>>}
	\label{templ:image4}
\end{figure}

\begin{figure}[ht]
	\centering
	\includegraphics[width=1\linewidth]{images/создатьтест}
	\caption{Композиция интерфейса сервиса <<Создание теста>>}
	\label{templ:image5}
\end{figure}

\begin{figure}[ht]
	\centering
	\includegraphics[width=1\linewidth]{images/результаты}
	\caption{Композиция интерфейса сервиса <<Результаты тестов>>}
	\label{templ:image6}
\end{figure}

\begin{figure}[ht]
	\centering
	\includegraphics[width=1\linewidth]{images/Авторизация}
	\caption{Композиция интерфейса сервиса <<Авторизация>>}
	\label{templ:image7}
\end{figure}


\begin{figure}[ht]
	\centering
	\includegraphics[width=1\linewidth]{images/профиль}
	\caption{Композиция интерфейса сервиса <<Профиль>>}
	\label{templ:image8}
\end{figure}


\begin{figure}[ht]
	\centering
	\includegraphics[width=1\linewidth]{images/создатькурс}
	\caption{Композиция интерфейса сервиса <<Создание курса>>}
	\label{templ:image9}
\end{figure}


\begin{figure}[ht]
	\centering
	\includegraphics[width=1\linewidth]{images/создатьурок}
	\caption{Композиция интерфейса сервиса <<Создание урока>>}
	\label{templ:image10}
\end{figure}

\clearpage
\subsection{Моделирование вариантов использования}

Для разработки платформы была построена UML-диаграмма вариантов использования, отражающая взаимодействие пользователей с системой.

Основные прецеденты системы:
\begin{enumerate}
\item Авторизация и выход из системы.
\item Управление курсами.
\item Управление уроками.
\item Создание и управление тестами.
\item Прохождение тестов.
\item Просмотр результатов тестов.
\item Локализация интерфейса.

\end{enumerate}

\subsection{Требования к оформлению документации}

Разработка программной документации и программного изделия должна производиться согласно ГОСТ 19.102–77 и ГОСТ 34.601–90. Единая система программной документации.

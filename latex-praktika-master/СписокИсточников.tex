\addcontentsline{toc}{section}{СПИСОК ИСПОЛЬЗОВАННЫХ ИСТОЧНИКОВ}

\begin{thebibliography}{9}

    \bibitem{javascript5} Бейли, Р. Методика преподавания программирования иностранным студентам / Р. Бейли. – Москва: Лань, 2021. – 210 с. – ISBN 978-5-8114-5678-9. – Текст~: непосредственный.
    
    \bibitem{e_learning} Бессонов, С. А. Электронное обучение: теория, методика, технологии / С. А. Бессонов. – Москва: Юрайт, 2021. – 356 с. – ISBN 978-5-534-12570-3. – Текст~: непосредственный.
    
    \bibitem{javascript1} Браун, И. Изучаем JavaScript: руководство по созданию современных веб-сайтов / И. Браун. – Москва: Эксмо, 2022. – 480 с. – ISBN 978-5-04-123456-7. – Текст~: непосредственный.
    
    \bibitem{css1} Веру, Л. CSS-секреты. 47 советов по улучшению веб-интерфейсов / Л. Веру. – Санкт-Петербург: Питер, 2017. – 368 с. – ISBN 978-5-496-02699-4. – Текст~: непосредственный.
    
    \bibitem{html1}	Голдстайн, А. HTML5 и CSS3 для всех / А. Голдстайн, Л. Лазарис, Э. Уэйл. – Москва~: Вильямс, 2012. – 368 с. – ISBN 978-5-699-57580-0. – Текст~: непосредственный.
    
    \bibitem{python_web} Григгс, Э. Python и Django. Разработка и развёртывание веб-приложений / Э. Григгс. – Москва: ДМК Пресс, 2020. – 480 с. – ISBN 978-5-97060-744-9. – Текст~: непосредственный.
    
    \bibitem{htmlcss} Дэкетт, Д. HTML и CSS. Разработка и создание веб-сайтов / Д. Дэкетт. – Москва: Эксмо, 2014. – 480 с. – ISBN 978-5-699-64193-2. – Текст~: непосредственный.
	
    \bibitem{javascript2} Кантор, Д. JavaScript для начинающих: от основ до веб-приложений / Д. Кантор. – Санкт-Петербург: Питер, 2023. – 360 с. – ISBN 978-5-4461-2345-2. – Текст~: непосредственный.
    
    \bibitem{sqlite} Кинг, Д. SQLite. Руководство разработчика / Д. Кинг. – Москва: Вильямс, 2021. – 256 с. – ISBN 978-5-8459-1876-3. – Текст~: непосредственный.
    
    \bibitem{js3} Кларк, А. Технологии адаптивного обучения в IT-образовании / А. Кларк. – Санкт-Петербург: БХВ, 2022. – 180 с. – ISBN 978-5-9775-0987-6. – Текст~: непосредственный.
    
    \bibitem{architecture} Лапшин, А. А. Архитектура информационных систем / А. А. Лапшин. – Москва: Юрайт, 2023. – 440 с. – ISBN 978-5-534-12750-9. – Текст~: непосредственный.
    
    \bibitem{software_design} Ларман, К. Приём проектирования программного обеспечения на основе UML / К. Ларман. – Москва: ДМК Пресс, 2020. – 768 с. – ISBN 978-5-97060-783-8. – Текст~: непосредственный.
    
    \bibitem{javascript3} Макфарланд, Д. JavaScript и jQuery: интерактивная фронтенд-разработка / Д. Макфарланд. – Москва: Диалектика, 2021. – 512 с. – ISBN 978-5-907203-45-6. – Текст: непосредственный.  

    \bibitem{python1} Матей, Н. Python. К вершинам мастерства / Н. Матей. – Санкт-Петербург: Питер, 2022. – 432 с. – ISBN 978-5-4461-0926-5. – Текст~: непосредственный.
    
    \bibitem{ajax} Пауэлл, Т. JavaScript. Полное руководство / Т. Пауэлл. – Санкт-Петербург: Символ-Плюс, 2021. – 896 с. – ISBN 978-5-93286-388-0. – Текст~: непосредственный.
	
	\bibitem{python} Петров, И. П. Python для веб-разработки: от основ до продвинутых техник / И. П. Петров. – Санкт-Петербург: Питер, 2024. – 416 с. – ISBN 978-5-4461-1789-5. – Текст~: непосредственный.
	
    \bibitem{javascript4} Симпсон, К. Глубокая работа с JavaScript: замыкания, прототипы, асинхронность / К. Симпсон. – Санкт-Петербург: Питер, 2022. – 288 с. – ISBN 978-5-4461-1987-5. – Текст~: непосредственный.  
    
    \bibitem{testing} Соммервилл, И. Инженерия программного обеспечения / И. Соммервилл. – Москва: Вильямс, 2021. – 864 с. – ISBN 978-5-8459-1932-6. – Текст~: непосредственный.
	
	\bibitem{html2}	Титтел, Э. HTML5 и CSS3 для чайников / Э. Титтел, К. Минник. – Москва~: Вильямс, 2016 – 400 с. – ISBN 978-1-118-65720-1. – Текст~: непосредственный.
	
	\bibitem{uml} Фаулер, М. UML. Основы / М. Фаулер. – Москва: ДМК Пресс, 2020. – 336 с. – ISBN 978-5-97060-681-7. – Текст~: непосредственный.
	
	\bibitem{frontend} Фрил, А. HTML5 и CSS3. Разработка интерактивных веб-сайтов / А. Фрил. – Санкт-Петербург: Питер, 2022. – 640 с. – ISBN 978-5-496-03720-4. – Текст~: непосредственный.
	
	\bibitem{web_applications} Харрисон, К. Разработка веб-приложений на Python / К. Харрисон. – Санкт-Петербург: Питер, 2021. – 400 с. – ISBN 978-5-4461-1860-1. – Текст~: непосредственный.
	
	\bibitem{django}	Холл, Б. Django. Разработка веб-приложений / Б. Холл.– Москва: ДМК Пресс, 2023.– 512 с. – ISBN 978-5-97060-789-0.– Текст~: непосредственный.
	
	\bibitem{dbdesign} Элмасри, Р., Наватхе, Ш. Системы баз данных. Проектирование, реализация и сопровождение / Р. Элмасри, Ш. Наватхе. – Москва: Вильямс, 2019. – 1184 с. – ISBN 978-5-8459-1959-3. – Текст~: непосредственный.
	
    \bibitem{css2} Юэнс, В. CSS. Карманный справочник / В. Юэнс. – Москва: Символ-Плюс, 2020. – 272 с. – ISBN 978-5-93286-362-0. – Текст~: непосредственный.


\end{thebibliography}

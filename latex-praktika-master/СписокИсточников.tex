\addcontentsline{toc}{section}{СПИСОК ИСПОЛЬЗОВАННЫХ ИСТОЧНИКОВ}

\begin{thebibliography}{9}

    \bibitem{javascript5} Бейли, Р. Методика преподавания программирования иностранным студентам / Р. Бейли. – Москва: Лань, 2021. – 210 с. – ISBN 978-5-8114-5678-9. – Текст~: непосредственный.
    
    \bibitem{javascript1} Браун, И. Изучаем JavaScript: руководство по созданию современных веб-сайтов / И. Браун. – Москва: Эксмо, 2022. – 480 с. – ISBN 978-5-04-123456-7. – Текст~: непосредственный.
    
    \bibitem{css1} Веру, Л. CSS-секреты. 47 советов по улучшению веб-интерфейсов / Л. Веру. – Санкт-Петербург: Питер, 2017. – 368 с. – ISBN 978-5-496-02699-4. – Текст~: непосредственный.
    
    \bibitem{htmlcss} Дэкетт, Д. HTML и CSS. Разработка и создание веб-сайтов / Д. Дэкетт. – Москва: Эксмо, 2014. – 480 с. – ISBN 978-5-699-64193-2. – Текст~: непосредственный.

    \bibitem{javascript2} Кантор, Д. JavaScript для начинающих: от основ до веб-приложений / Д. Кантор. – Санкт-Петербург: Питер, 2023. – 360 с. – ISBN 978-5-4461-2345-2. – Текст~: непосредственный.
    
    \bibitem{js3} Кларк, А. Технологии адаптивного обучения в IT-образовании / А. Кларк. – Санкт-Петербург: БХВ, 2022. – 180 с. – ISBN 978-5-9775-0987-6. – Текст~: непосредственный.
    
    \bibitem{javascript3} Макфарланд, Д. JavaScript и jQuery: интерактивная фронтенд-разработка / Д. Макфарланд. – Москва: Диалектика, 2021. – 512 с. – ISBN 978-5-907203-45-6. – Текст: непосредственный.  

    \bibitem{python1} Матей, Н. Python. К вершинам мастерства / Н. Матей. – Санкт-Петербург: Питер, 2022. – 432 с. – ISBN 978-5-4461-0926-5. – Текст~: непосредственный.

    \bibitem{javascript4} Симпсон, К. Глубокая работа с JavaScript: замыкания, прототипы, асинхронность / К. Симпсон. – Санкт-Петербург: Питер, 2022. – 288 с. – ISBN 978-5-4461-1987-5. – Текст~: непосредственный.  

    \bibitem{css2} Юэнс, В. CSS. Карманный справочник / В. Юэнс. – Москва: Символ-Плюс, 2020. – 272 с. – ISBN 978-5-93286-362-0. – Текст~: непосредственный.


\end{thebibliography}

\section*{ВВЕДЕНИЕ}
\addcontentsline{toc}{section}{ВВЕДЕНИЕ}

Современные информационные технологии играют ключевую роль в образовательном процессе, обеспечивая доступ к учебным материалам и поддерживая самостоятельное обучение. Обучающие платформы становятся важным инструментом для изучения программирования, особенно для иностранных студентов, которым требуется гибкий и доступный формат обучения с учетом языковых особенностей. Язык программирования JavaScript является одним из наиболее востребованных в веб-разработке благодаря своей универсальности и широкому применению в создании интерактивных веб-приложений. Разработка специализированных веб-приложений для обучения JavaScript позволяет эффективно организовать процесс самостоятельной работы студентов, предоставляя структурированные курсы, интерактивные тесты и отслеживание прогресса.

Веб-приложение, ориентированное на иностранных студентов, предполагает использование мультиязычного интерфейса и интерактивных инструментов, таких как редакторы контента и анимации, для повышения вовлеченности и удобства обучения. Такие платформы позволяют студентам изучать JavaScript в удобном темпе, а преподавателям — эффективно управлять образовательным контентом и отслеживать результаты.

Создание веб-приложения для обучения программированию способствует не только повышению квалификации студентов, но и расширению доступности образования для международной аудитории. Платформа становится инструментом, который объединяет образовательные ресурсы, тестирование и аналитику в единой цифровой среде, упрощая процесс обучения и делая его более интерактивным.

Цель настоящей работы – разработка веб-приложения для компьютерной поддержки самостоятельной работы иностранных студентов при изучении языка программирования JavaScript, обеспечивающего доступ к образовательным материалам, тестированию и отслеживанию прогресса через мультиязычный интерфейс для повышения эффективности обучения. Для достижения поставленной цели необходимо решить следующие задачи:
\begin{itemize}
	\item провести анализ предметной области и требований к обучающим платформам для иностранных студентов;
	\item разработать концептуальную модель веб-приложения для обучения JavaScript;
	\item спроектировать веб-приложение с учетом мультиязычности и интерактивности;
	\item реализовать веб-приложение средствами современных веб-технологий.
\end{itemize}

\emph{Структура и объем работы.} Отчет состоит из введения, 4 разделов основной части, заключения, списка использованных источников, 2 приложений. Текст выпускной квалификационной работы равен \formbytotal{lastpage}{страниц}{е}{ам}{ам}.

\emph{Во введении} сформулирована цель работы, поставлены задачи разработки, описана структура работы, приведено краткое содержание каждого из разделов.

\emph{В первом разделе} на стадии анализа предметной области приводится сбор информации о требованиях к обучающим платформam's для иностранных студентов, изучающих JavaScript, и особенностях организации самостоятельной работы.

\emph{Во втором разделе} на стадии технического задания приводятся требования к разрабатываемому веб-приложению, включая поддержку мультиязычности и интерактивных элементов.

\emph{В третьем разделе} на стадии технического проектирования представлены проектные решения для веб-приложения, включая модели данных и структуру интерфейса.

\emph{В четвертом разделе} приводится список классов и их методов, использованных при разработке платформы, а также результаты тестирования разработанного веб-приложения.

В заключении излагаются основные результаты работы, полученные в ходе разработки.

В приложении А представлен графический материал.
В приложении Б представлены фрагменты исходного кода. 

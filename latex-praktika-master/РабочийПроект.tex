\section{Рабочий проект}

\subsection{Спецификация класса Course}

Модель Course хранит информацию о курсе, созданном преподавателем.

Описание: организует уроки и тесты, поддерживает мультиязычные названия и описания.

Валидация данных:
	\begin{itemize}
		\item Поля title, description валидируются через CourseForm;
		\item Поле image проверяется на формат (JPEG, PNG).
	\end{itemize}
	
Связи:
	\begin{itemize}
		\item внешний ключ creator на User;
		\item связан с Lesson, Enrollment.
	\end{itemize}
		
Безопасность:
	\begin{itemize}
		\item редактирование доступно только creator через CourseUpdateView;
		\item студенты видят только курсы с isactive=True.
	\end{itemize}
	
Уведомления:
	\begin{itemize}
		\item уведомления через Django messages (например, «Курс создан»).
	\end{itemize}
	
Основные методы:
	\begin{itemize}
		\item str(): возвращает title.
		\item getabsoluteurl(): возвращает URL курса;
		\item save(): обрабатывает изображение.
	\end{itemize}
	
Зависимости:
	\begin{itemize}
		\item связан с Lesson, Test, Enrollment;
		\item использует CKEditor для description.
	\end{itemize}


\begin{xltabular}{\textwidth}{|l|l|l|X|}
	\caption{Данные класса Course\label{tab:course_attributes}}\\
	\hline
	Поле & Тип & Обязательное & Описание \\ \hline
	\endfirsthead
	\continuecaption{Продолжение таблицы \ref{tab:course_attributes}}\\
	\hline
	Поле & Тип & Обязательное & Описание \\ \hline
	\endhead
	id & Integer & да & Уникальный идентификатор \\ \hline
	title & String & да & Название курса \\ \hline
	description & Text & нет & Описание (HTML, CKEditor) \\ \hline
	image & Image & нет & Изображение курса \\ \hline
	isactive & Boolean & да & Статус публикации \\ \hline
	creator & ForeignKey & да & Преподаватель (User) \\ \hline
\end{xltabular}

\begin{xltabular}{\textwidth}{|l|X|}
	\caption{Методы класса Course\label{tab:course_methods}}\\
	\hline
	Метод & Описание \\ \hline
	\endfirsthead
	\continuecaption{Продолжение таблицы \ref{tab:course_methods}}\\
	\hline
	Метод & Описание \\ \hline
	\endhead
	str() & Возвращает title \\ \hline
	getabsoluteurl() & Возвращает URL курса \\ \hline
	save() & Обрабатывает изображение \\ \hline
\end{xltabular}

\begin{lstlisting}[language=Python, caption=Представление для создания курса, label=lst:course_view]
	from django.contrib.auth.decorators import login_required
	from django.shortcuts import render, redirect
	from .models import Course
	from .forms import CourseForm
	
	@login_required
	def create_course(request):
	if not request.user.is_teacher:
	return redirect('home')
	if request.method == 'POST':
	form = CourseForm(request.POST, request.FILES)
	if form.is_valid():
	course = form.save(commit=False)
	course.creator = request.user
	course.save()
	return redirect('courses')
	else:
	form = CourseForm()
	return render(request, 'courses/create.html', {'form': form})
\end{lstlisting}

\subsubsection{Спецификация класса Lesson}

Модель Lesson описывает урок в составе курса.

Описание: хранит контент, видео и упражнения, поддерживает мультиязычность.

Валидация данных:
	\begin{itemize}
		\item поля title, content валидируются через LessonForm;
		\item поле videourl проверяется на корректность.
	\end{itemize}
	
Связи:
	\begin{itemize}
		\item внешний ключ course на Course;
		\item связан с Test, StudentProgress.
	\end{itemize}
	
Безопасность:
	\begin{itemize}
		\item редактирование ограничено creator курса;
		\item доступ для студентов: ispublished=True.
	\end{itemize}
	
Уведомления:
	\begin{itemize}
		\item уведомления через Django messages (например, «Урок добавлен»).
	\end{itemize}
	
Основные методы:
	\begin{itemize}
		\item str(): возвращает title;
		\item save(): проверяет order.
	\end{itemize}
	
Зависимости:
	\begin{itemize}
		\item связан с Course, Test, StudentProgress;
		\item использует CKEditor для content.
	\end{itemize}


\begin{xltabular}{\textwidth}{|l|l|l|X|}
	\caption{Данные класса Lesson\label{tab:lesson_attributes}}\\
	\hline
	Поле & Тип & Обязательное & Описание \\ \hline
	\endfirsthead
	\continuecaption{Продолжение таблицы \ref{tab:lesson_attributes}}\\
	\hline
	Поле & Тип & Обязательное & Описание \\ \hline
	\endhead
	id & Integer & да & Уникальный идентификатор \\ \hline
	title & String & да & Название урока \\ \hline
	content & Text & да & Содержимое (HTML, CKEditor) \\ \hline
	videourl & URL & нет & Ссылка на видео \\ \hline
	order & Integer & да & Порядок урока \\ \hline
	ispublished & Boolean & да & Статус публикации \\ \hline
	exercise & Text & нет & Интерактивное упражнение \\ \hline
	expectedresult & Text & нет & Ожидаемый результат \\ \hline
	course & ForeignKey & да & Курс (Course) \\ \hline
\end{xltabular}

\begin{xltabular}{\textwidth}{|l|X|}
	\caption{Методы класса Lesson\label{tab:lesson_methods}}\\
	\hline
	Метод & Описание \\ \hline
	\endfirsthead
	\continuecaption{Продолжение таблицы \ref{tab:lesson_methods}}\\
	\hline
	Метод & Описание \\ \hline
	\endhead
	str() & Возвращает title \\ \hline
	save() & Проверяет order \\ \hline
\end{xltabular}

\subsubsection{Спецификация класса Test}

Модель Test описывает тест, связанный с уроком.

Описание: проверяет знания студентов, поддерживает мультиязычные вопросы.

Валидация данных:
	\begin{itemize}
		\item поля title, description валидируются через TestForm;
		\item поле passingscore — положительное значение.
	\end{itemize}
	
Связи:
	\begin{itemize}
		\item внешний ключ lesson на Lesson;
		\item связан с Question, TestResult;
	\end{itemize}
	
Безопасность:
	\begin{itemize}
		\item редактирование ограничено creator урока;
		\item доступ для студентов: isactive=True;
	\end{itemize}
	
Уведомления:
	\begin{itemize}
		\item уведомления через Django messages (например, «Тест создан»);
	\end{itemize}
	
Основные методы:
	\begin{itemize}
		\item str(): возвращает title;
		\item save(): проверяет passingscore.
	\end{itemize}
	
Зависимости:
	\begin{itemize}
		\item связан с Lesson, Question, TestResult;
		\item используется в testform.html.
	\end{itemize}


\begin{xltabular}{\textwidth}{|l|l|l|X|}
	\caption{Данные класса Test\label{tab:test_attributes}}\\
	\hline
	Поле & Тип & Обязательное & Описание \\ \hline
	\endfirsthead
	\continuecaption{Продолжение таблицы \ref{tab:test_attributes}}\\
	\hline
	Поле & Тип & Обязательное & Описание \\ \hline
	\endhead
	id & Integer & да & Уникальный идентификатор \\ \hline
	title & String & да & Название теста \\ \hline
	description & Text & нет & Описание теста \\ \hline
	passingscore & Integer & да & Проходной балл \\ \hline
	isactive & Boolean & да & Статус активности \\ \hline
	lesson & ForeignKey & да & Урок (Lesson) \\ \hline
\end{xltabular}

\begin{xltabular}{\textwidth}{|l|X|}
	\caption{Методы класса Test\label{tab:test_methods}}\\
	\hline
	Метод & Описание \\ \hline
	\endfirsthead
	\continuecaption{Продолжение таблицы \ref{tab:test_methods}}\\
	\hline
	Метод & Описание \\ \hline
	\endhead
	str() & Возвращает title \\ \hline
	save() & Проверяет passingscore \\ \hline
\end{xltabular}

\subsubsection{Спецификация класса Question}

Модель Question описывает вопрос в тесте.


Описание: поддерживает различные типы вопросов (одиночный, множественный, текстовый) с мультиязычным текстом.

Валидация данных:
	\begin{itemize}
		\item поле text валидируется через QuestionForm;
		\item поле questiontype принимает значения single, multiple, text;
		\item поле points проверяется на положительное значение.
	\end{itemize}
	
Связи:
	\begin{itemize}
		\item внешний ключ test на Test;
		\item связан с Answer.
	\end{itemize}
	
Безопасность:
	\begin{itemize}
		\item редактирование ограничено creator урока через представления;
		\item доступ для студентов только через активные тесты.
	\end{itemize}
	
Уведомления:
	\begin{itemize}
		\item уведомления через Django messages (например, «Вопрос добавлен»).
	\end{itemize}
	
Основные методы:
	\begin{itemize}
		\item str(): возвращает text;
		\item save(): проверяет questiontype и points.
	\end{itemize}
	
Зависимости:
	\begin{itemize}
		\item связан с Test, Answer;
		\item используется в шаблоне questionform.html.
	\end{itemize}


\begin{xltabular}{\textwidth}{|l|l|l|X|}
	\caption{Данные класса Question\label{tab:question_attributes}}\\
	\hline
	Поле & Тип & Обязательное & Описание \\ \hline
	\endfirsthead
	\continuecaption{Продолжение таблицы \ref{tab:question_attributes}}\\
	\hline
	Поле & Тип & Обязательное & Описание \\ \hline
	\endhead
	id & Integer & да & Уникальный идентификатор \\ \hline
	text & Text & да & Текст вопроса \\ \hline
	questiontype & String & да & Тип вопроса (single, multiple, text) \\ \hline
	points & Integer & да & Баллы за ответ \\ \hline
	test & ForeignKey & да & Тест (Test) \\ \hline
\end{xltabular}

\begin{xltabular}{\textwidth}{|l|X|}
	\caption{Методы класса Question\label{tab:question_methods}}\\
	\hline
	Метод & Описание \\ \hline
	\endfirsthead
	\continuecaption{Продолжение таблицы \ref{tab:question_methods}}\\
	\hline
	Метод & Описание \\ \hline
	\endhead
	str() & Возвращает text \\ \hline
	save() & Проверяет questiontype и points \\ \hline
\end{xltabular}

\subsubsection{Спецификация класса Answer}

Модель Answer описывает вариант ответа на вопрос в тесте.

Описание: хранит текст ответа и флаг правильности, поддерживает мультиязычность.

Валидация данных:
	\begin{itemize}
		\item поле text валидируется через AnswerForm;
		\item поле iscorrect — булево, проверяется на уровне формы;
		\item поле order проверяется на положительное значение.
	\end{itemize}
	
Связи:
	\begin{itemize}
		\item внешний ключ question на Question;
		\item используется в TestResult.
	\end{itemize}
	
Безопасность:
	\begin{itemize}
		\item редактирование ограничено creator урока через AnswerCreateView;
		\item доступ для студентов только через активные тесты.
	\end{itemize}
	
Уведомления:
	\begin{itemize}
		\item уведомления через Django messages (например, «Ответ добавлен»).
	\end{itemize}
	
Лсновные методы:
	\begin{itemize}
		\item str(): возвращает text;
		\item save(): проверяет iscorrect и order.
	\end{itemize}
	
Зависимости:
	\begin{itemize}
		\item связан с Question, TestResult;
		\item используется в шаблоне answerform.html.
	\end{itemize}


\begin{xltabular}{\textwidth}{|l|l|l|X|}
	\caption{Данные класса Answer\label{tab:answer_attributes}}\\
	\hline
	Поле & Тип & Обязательное & Описание \\ \hline
	\endfirsthead
	\continuecaption{Продолжение таблицы \ref{tab:answer_attributes}}\\
	\hline
	Поле & Тип & Обязательное & Описание \\ \hline
	\endhead
	id & Integer & да & Уникальный идентификатор \\ \hline
	text & Text & да & Текст ответа \\ \hline
	iscorrect & Boolean & да & Признак правильности \\ \hline
	order & Integer & да & Порядок отображения \\ \hline
	question & ForeignKey & да & Вопрос (Question) \\ \hline
\end{xltabular}

\begin{xltabular}{\textwidth}{|l|X|}
	\caption{Методы класса Answer\label{tab:answer_methods}}\\
	\hline
	Метод & Описание \\ \hline
	\endfirsthead
	\continuecaption{Продолжение таблицы \ref{tab:answer_methods}}\\
	\hline
	Метод & Описание \\ \hline
	\endhead
	str() & Возвращает text \\ \hline
	save() & Проверяет iscorrect и order \\ \hline
\end{xltabular}

\subsubsection{Спецификация класса TestResult}

Модель TestResult хранит результаты прохождения тестов студентами.


Описание: фиксирует баллы, выбранные ответы и дату завершения.

Валидация данных:
	\begin{itemize}
		\item поле score проверяется на положительное значение и соответствие максимуму теста;
		\item поле answers валидируется через TestSubmissionForm.
	\end{itemize}
	
Связи:
	\begin{itemize}
		\item внешний ключ student на User;
		\item внешний ключ lesson на Lesson;
		\item связь ManyToMany с Answer.
	\end{itemize}
	
Безопасность:
	\begin{itemize}
		\item доступ ограничен student или creator урока через представления;
		\item результаты видны только авторизованным пользователям.
	\end{itemize}
	
Уведомления:
	\begin{itemize}
		\item уведомления через Django messages (например, «Тест успешно пройден»).
	\end{itemize}
	
Основные методы:
	\begin{itemize}
		\item str(): возвращает информацию о результате (например, «Результат студента X по уроку Y»);
		\item save(): пересчитывает score на основе answers.
	\end{itemize}
	
Зависимости:
	\begin{itemize}
		\item связан с User, Lesson, Answer;
		\item используется в шаблоне testresult.html.
	\end{itemize}


\begin{xltabular}{\textwidth}{|l|l|l|X|}
	\caption{Данные класса TestResult\label{tab:testresult_attributes}}\\
	\hline
	Поле & Тип & Обязательное & Описание \\ \hline
	\endfirsthead
	\continuecaption{Продолжение таблицы \ref{tab:testresult_attributes}}\\
	\hline
	Поле & Тип & Обязательное & Описание \\ \hline
	\endhead
	id & Integer & да & Уникальный идентификатор \\ \hline
	student & ForeignKey & да & Студент (User) \\ \hline
	lesson & ForeignKey & да & Урок (Lesson) \\ \hline
	score & Integer & да & Полученные баллы \\ \hline
	answers & ManyToMany & да & Выбранные ответы \\ \hline
	attempts & Integer & да & Количество попыток \\ \hline
	completedat & DateTime & да & Дата завершения \\ \hline
\end{xltabular}

\begin{xltabular}{\textwidth}{|l|X|}
	\caption{Методы класса TestResult\label{tab:testresult_methods}}\\
	\hline
	Метод & Описание \\ \hline
	\endfirsthead
	\continuecaption{Продолжение таблицы \ref{tab:testresult_methods}}\\
	\hline
	Метод & Описание \\ \hline
	\endhead
	str() & Возвращает информацию о результате \\ \hline
	save() & Пересчитывает score \\ \hline
\end{xltabular}

\subsubsection{Спецификация класса Enrollment}

Модель Enrollment фиксирует запись студента на курс.


Описание: связывает студента и курс для отслеживания участия.

Валидация данных:
	\begin{itemize}
		\item поля student и course проверяются на уникальность через uniquetogether.
	\end{itemize}
	
Связи:
	\begin{itemize}
		\item внешний ключ student на User;
		\item внешний ключ course на Course.
	\end{itemize}
	
Безопасность:
	\begin{itemize}
		\item доступ ограничен student или преподавателю через представления;
		\item преподаватели видят записи через teacherdashboard.
	\end{itemize}
	
Уведомления:
	\begin{itemize}
		\item уведомления через Django messages (например, «Вы записаны на курс»).
	\end{itemize}
	
Основные методы:
	\begin{itemize}
		\item str(): возвращает информацию о записи (например, «Студент X записан на курс Y»);
		\item save(): проверяет уникальность записи.
	\end{itemize}
	
Зависимости:
	\begin{itemize}
		\item связан с User, Course;
		\item используется в представлениях enrollcourse, unenrollcourse.
	\end{itemize}


\begin{xltabular}{\textwidth}{|l|l|l|X|}
	\caption{Данные класса Enrollment\label{tab:enrollment_attributes}}\\
	\hline
	Поле & Тип & Обязательное & Описание \\ \hline
	\endfirsthead
	\continuecaption{Продолжение таблицы \ref{tab:enrollment_attributes}}\\
	\hline
	Поле & Тип & Обязательное & Описание \\ \hline
	\endhead
	id & Integer & да & Уникальный идентификатор \\ \hline
	student & ForeignKey & да & Студент (User) \\ \hline
	course & ForeignKey & да & Курс (Course) \\ \hline
\end{xltabular}

\begin{xltabular}{\textwidth}{|l|X|}
	\caption{Методы класса Enrollment\label{tab:enrollment_methods}}\\
	\hline
	Метод & Описание \\ \hline
	\endfirsthead
	\continuecaption{Продолжение таблицы \ref{tab:enrollment_methods}}\\
	\hline
	Метод & Описание \\ \hline
	\endhead
	str() & Возвращает информацию о записи \\ \hline
	save() & Проверяет уникальность \\ \hline
\end{xltabular}

\subsubsection{Спецификация класса StudentProgress}

Модель StudentProgress отслеживает прогресс студента по урокам.


Описание: фиксирует завершённость уроков и дату завершения.

Валидация данных:
	\begin{itemize}
		\item Поле completed — булево, устанавливается через представления;
		\item Поле completedat заполняется автоматически при completed=True.
	\end{itemize}
	
Связи:
	\begin{itemize}
		\item внешний ключ student на User;
		\item внешний ключ lesson на Lesson.
	\end{itemize}
	
Безопасность:
	\begin{itemize}
		\item доступ ограничен student через представления;
		\item преподаватели видят прогресс через teacherdashboard.
	\end{itemize}
	
Уведомления:
	\begin{itemize}
		\item уведомления через Django messages (например, «Урок завершён»).
	\end{itemize}
	
Основные методы:
	\begin{itemize}
		\item str(): возвращает информацию о прогрессе (например, «Прогресс студента X по уроку Y»);
		\item save(): устанавливает completedat.
	\end{itemize}
	
Зависимости:
	\begin{itemize}
		\item связан с User, Lesson;
		\item используется в представлениях completelesson, studentdashboard.
	\end{itemize}


\begin{xltabular}{\textwidth}{|l|l|l|X|}
	\caption{Данные класса StudentProgress\label{tab:studentprogress_attributes}}\\
	\hline
	Поле & Тип & Обязательное & Описание \\ \hline
	\endfirsthead
	\continuecaption{Продолжение таблицы \ref{tab:studentprogress_attributes}}\\
	\hline
	Поле & Тип & Обязательное & Описание \\ \hline
	\endhead
	id & Integer & да & Уникальный идентификатор \\ \hline
	student & ForeignKey & да & Студент (User) \\ \hline
	lesson & ForeignKey & да & Урок (Lesson) \\ \hline
	completed & Boolean & да & Завершённость урока \\ \hline
	completedat & DateTime & нет & Дата завершения \\ \hline
\end{xltabular}

\begin{xltabular}{\textwidth}{|l|X|}
	\caption{Методы класса StudentProgress\label{tab:studentprogress_methods}}\\
	\hline
	Метод & Описание \\ \hline
	\endfirsthead
	\continuecaption{Продолжение таблицы \ref{tab:studentprogress_methods}}\\
	\hline
	Метод & Описание \\ \hline
	\endhead
	str() & Возвращает информацию о прогрессе \\ \hline
	save() & Устанавливает completedat \\ \hline
\end{xltabular}

\subsubsection{Спецификация класса Achievement}

Модель Achievement описывает достижения, присваиваемые студентам.


Описание: хранит информацию о наградах (например, «Первый завершённый урок»)).

Валидация данных:
	\begin{itemize}
		\item поля title, description валидируются через AchievementForm;
		\item проверка уникальности title через Django ORM.
	\end{itemize}
	
Связи:
	\begin{itemize}
		\item связан с UserAchievement через внешний ключ.
	\end{itemize}
	
Безопасность:
	\begin{itemize}
		\item редактирование ограничено администраторам через isstaff.
	\end{itemize}
	
Уведомления:
	\begin{itemize}
		\item уведомления через Django messages (например, «Достижение создано»).
	\end{itemize}
	
Основные методы:
	\begin{itemize}
		\item str(): возвращает title;
		\item save(): проверяет уникальность title.
	\end{itemize}
	
Зависимости:
	\begin{itemize}
		\item связан с UserAchievement;
		\item используется в studentdashboard.
	\end{itemize}


\begin{xltabular}{\textwidth}{|l|l|l|X|}
	\caption{Данные класса Achievement\label{tab:achievement_attributes}}\\
	\hline
	Поле & Тип & Обязательное & Описание \\ \hline
	\endfirsthead
	\continuecaption{Продолжение таблицы \ref{tab:achievement_attributes}}\\
	\hline
	Поле & Тип & Обязательное & Описание \\ \hline
	\endhead
	id & Integer & да & Уникальный идентификатор \\ \hline
	title & String & да & Название достижения \\ \hline
	description & Text & нет & Описание достижения \\ \hline
\end{xltabular}

\begin{xltabular}{\textwidth}{|l|X|}
	\caption{Методы класса Achievement\label{tab:achievement_methods}}\\
	\hline
	Метод & Описание \\ \hline
	\endfirsthead
	\continuecaption{Продолжение таблицы \ref{tab:achievement_methods}}\\
	\hline
	Метод & Описание \\ \hline
	\endhead
	str() & Возвращает title \\ \hline
	save() & Проверяет уникальность title \\ \hline
\end{xltabular}

\subsubsection{Спецификация класса UserAchievement}

Модель UserAchievement связывает пользователей с достижениями.


Описание: отслеживает, какие достижения присвоены студентам.

Валидация данных:
	\begin{itemize}
		\item поля user и achievement проверяются на уникальность через uniquetogether.
	\end{itemize}
	
Связи:
	\begin{itemize}
		\item внешний ключ user на User;
		\item внешний ключ achievement на Achievement;
	\end{itemize}
	
Безопасность:
	\begin{itemize}
		\item доступ ограничен user через представления;
		\item преподаватели видят достижения через teacherdashboard.
	\end{itemize}
	
Уведомления:
	\begin{itemize}
		\item уведомления через Django messages (например, «Новое достижение получено»).
	\end{itemize}
	
Основные методы:
	\begin{itemize}
		\item str(): возвращает информацию о достижении (например, «Достижение X для пользователя Y»);
		\item save(): проверяет уникальность связи.
	\end{itemize}
	
Зависимости:
	\begin{itemize}
		\item связан с User, Achievement;
		\item используется в studentdashboard.
	\end{itemize}


\begin{xltabular}{\textwidth}{|l|l|l|X|}
	\caption{Данные класса UserAchievement\label{tab:userachievement_attributes}}\\
	\hline
	Поле & Тип & Обязательное & Описание \\ \hline
	\endfirsthead
	\continuecaption{Продолжение таблицы \ref{tab:userachievement_attributes}}\\
	\hline
	Поле & Тип & Обязательное & Описание \\ \hline
	\endhead
	id & Integer & да & Уникальный идентификатор \\ \hline
	user & ForeignKey & да & Пользователь (User) \\ \hline
	achievement & ForeignKey & да & Достижение (Achievement) \\ \hline
	awardedat & DateTime & да & Дата присвоения \\ \hline
\end{xltabular}

\begin{xltabular}{\textwidth}{|l|X|}
	\caption{Методы класса UserAchievement\label{tab:userachievement_methods}}\\
	\hline
	Метод & Описание \\ \hline
	\endfirsthead
	\continuecaption{Продолжение таблицы \ref{tab:userachievement_methods}}\\
	\hline
	Метод & Описание \\ \hline
	\endhead
	str() & Возвращает информацию о достижении \\ \hline
	save() & Проверяет уникальность связи \\ \hline
\end{xltabular}

\subsection{Модульное тестирование разработанного веб-приложения}

Модульное тестирование проводилось с использованием фреймворка django.test. Тестировались основные функции системы: регистрация пользователей, создание курсов, добавление уроков, прохождение тестов и локализация.

Тестируемые компоненты:
	\begin{enumerate}
		\item Модель User: проверка создания пользователей с ролями.
		\item Модель Course: проверка создания и валидации курсов.
		\item Модель Test: проверка подсчёта баллов.
		\item Локализация: переключение языков (ru/en).
	\end{enumerate}
Инструменты: django.test.TestCase, unittest.
Результаты: все тесты выполнены успешно.


\begin{lstlisting}[language=Python, caption=Модульный тест для User и Course, label=lst:user_course_test]
	from django.test import TestCase
	from django.contrib.auth import get_user_model
	from .models import Course
	
	User = get_user_model()
	
	class EducationPlatformTestCases(TestCase):
	def setUp(self):
	self.user = User.objects.create_user(username='teststudent', password='testpass123', is_student=True)
	self.course = Course.objects.create(title='JavaScript Basics', description='Introduction to JavaScript', creator=self.user)
	
	def test_user_and_course(self):
	self.assertEqual(self.user.username, 'teststudent')
	self.assertTrue(self.user.is_student)
	self.assertEqual(self.course.title, 'JavaScript Basics')
\end{lstlisting}

\begin{xltabular}{\textwidth}{|X|l|X|}
	\caption{Результаты модульного тестирования\label{tab:module_testing_results}}\\
	\hline
	Тест & Статус & Описание \\ \hline
	\endfirsthead
	\continuecaption{Продолжение таблицы \ref{tab:module_testing_results}}\\
	\hline
	Тест & Статус & Описание \\ \hline
	\endhead
	Регистрация пользователя & Успешно & Проверена корректность создания пользователя с ролями \\ \hline
	Создание курса & Успешно & Проверено создание курса преподавателем \\ \hline
	Добавление урока & Успешно & Проверена сортировка уроков по order \\ \hline
	Прохождение теста & Успешно & Проверен подсчёт баллов \\ \hline
	Локализация интерфейса & Успешно & Проверено переключение языков (ru/en) \\ \hline
\end{xltabular}

\subsection{Системное тестирование разработанного веб-приложения}

Системное тестирование проводилось для проверки взаимодействия компонентов и интерфейса.

Сценарии тестирования:
	\begin{enumerate}
		\item Регистрация и авторизация пользователя.
		\item Создание курса и запись студента.
		\item Добавление урока с контентом через CKEditor.
		\item Прохождение теста и просмотр результатов.
		\item Переключение языка интерфейса.
	\end{enumerate}
Результаты: все сценарии выполнены успешно;


\begin{xltabular}{\textwidth}{|X|l|X|}
	\caption{Результаты системного тестирования\label{tab:system_testing_results}}\\
	\hline
	Сценарий & Статус & Описание \\ \hline
	\endfirsthead
	\continuecaption{Продолжение таблицы \ref{tab:system_testing_results}}\\
	\hline
	Сценарий & Статус & Описание \\ \hline
	\endhead
	Регистрация и авторизация & Успешно & Пользователи регистрируются и входят \\ \hline
	Создание и запись на курс & Успешно & Курс создаётся, студенты записываются \\ \hline
	Добавление урока & Успешно & Урок добавляется, контент отображается \\ \hline
	Прохождение теста & Успешно & Тесты проходятся, результаты сохраняются \\ \hline
	Локализация & Успешно & Интерфейс переключается между языками \\ \hline
\end{xltabular}


\subsection{Сборка компонентов системы}

Веб-приложение построено по архитектуре MVC, реализованной в Django.

Компоненты:
\begin{enumerate}
		\item Модели: описаны выше, обеспечивают работу с данными через ORM.
		\item Представления: функции с декораторами loginrequired, teacherrequired.
		\item Шаблоны: используют Bootstrap, CKEditor, AOS.
		\item Маршруты: определены в urls.py.
\end{enumerate}
		
Настройка:
\begin{itemize}
		\item установка Python, Django, SQLite/PostgreSQL;
		\item настройка CKEditor, AOS, Bootstrap;
		\item локализация через django.utils.translation.
\end{itemize}

\begin{lstlisting}[language=Python, caption=Маршруты приложения, label=lst:urls]
	from django.urls import path
	from . import views
	
	urlpatterns = [
	path('courses/', views.course_list, name='courses'),
	path('courses/create/', views.create_course, name='create_course'),
	path('courses/<int:pk>/', views.course_detail, name='course_detail'),
	]
\end{lstlisting}